Ziel des Projektes ist es ein System zu entwerfen um Bus bunching (Pulkbildung) im �ffentlichen Nahverkehr, genauer bei den Bussen der Berliner Verkehrsbetrieben, zu verhindern.

Bus bunching ist ein Ph�nomen im �ffentlichen Personennahverkehr bei dem es durch Beeintr�chtigungen im Verkehrsfluss zu einer zusammenh�ngenden Fahrzeugreihe kommt. Daraus resultieren starke Schwankungen in den Folgezeiten der Busse einer Linie welche wiederum eine unregelm��ige Belastung der Fahrzeuge mit Fahrg�sten nach sich zieht.

Die hier angestrebt L�sung besteht darin, jedem Busfahrer eine Android Applikation zur verf�gung zu stellen, auf der der relative Zeitabstand zu Bussen direkt vor und hinter ihm visualisiert wird. Der eigene Bus wird durch eine Fahrtennummer im Vorfeld festgelegt. Der Fahrer kann, entsprechend der angezeigten Informationen, sein Fahrverhalten so anpassen das der Takt zwischen den Fahrzeugen den Vorgaben der Leitzentrale entspricht. Er kann beispielsweise, wenn der Bus vor ihm durch eine Beeintr�chtigung auf der Strecke versp�tet an einer Haltestelle ankommt und sich somit der relative Zeitabstand verringert, langsamer fahren oder an der n�chsten Haltestelle so lange warten bis der Zeitabstand wieder der Strecken Taktung entspricht.

Grundlage des Systems ist das Senden der Koordinaten jedes Buses aus der App heraus. Diese Koordinaten und die Fahrtdauer zwischen den Koordinaten werden in einer Datenbank gespeichert. Aus der Menge der Datens�tze einer Route ist es m�glich durchschnittliche Fahrzeiten f�r die Route zu unterschiedlichen Tageszeiten zu berechnen. Diese Berechnungen erlauben eine Vorhersage der relativen zeitlichen Distanz zwischen zwei Bussen einer Linie zu berechnen.