\documentclass[11pt,oneside,a4paper]{scrreprt}

\usepackage[ngerman]{babel}
\usepackage[latin1]{inputenc}
\usepackage[T1]{fontenc}
\usepackage[babel,german=quotes]{csquotes}
\usepackage{amssymb}
\usepackage{amsmath}
\usepackage{graphicx}
\usepackage{wrapfig}
\usepackage{color}
\usepackage{palatino}
\usepackage{float}
\usepackage{subfigure} 
\usepackage{tikz}

\begin{document}

% Deckplatt
\begin{titlepage}
\centering
\includegraphics{res/logo}\par\vspace{2cm}
{\scshape\LARGE Hochschule f�r Technik und Wirtschaft \par}
\vspace{1cm}
{\scshape\Large Projektarrbeit\par}
\vspace{3cm}
{\huge\bfseries Visualisierung  und Auswertung von Positionsdaten der Omnibusse der BVG\par}
\vspace{1.5cm}
{\Large Technik mobiler Systeme\par}
{\Large Ausgew�hlte Kapitel mobiler Anwendungen\par}
\vspace{2cm}
{\Large\itshape Pascal Dettmers (551733)\par}
{\Large\itshape Stefan Neuberger (553849)\par}
{\Large\itshape Tobias Ullerich (553746)\par}
\vfill

{\large \today\par}
\end{titlepage}

% Abbildungs- und Tabellenverzeichnis
\tableofcontents

\listoffigures
\begingroup
\let\clearpage\relax
\listoftables
\endgroup

% Inhalt
\chapter{Dokumentengeschichte}
\begin{table}[h]
	\begin{tabular}{|l|l|p{5,5cm}|}
            	\hline
            	Zeitraum & TPL/Autor(en) & �nderungen \\
            	\hline
		\textbf{Wintersemester 2017/18 (09.11.2017)} & Tobias Ullerich & 
			Initiale Dokumentenstruktur \newline Entwurf Aufgabenstellung \newline \\
         	\hline
		\textbf{Wintersemester 2017/18 (01.12.2017)} & Tobias Ullerich & 
			Analyse BVG Datenbank 1/2 \newline \\
         	\hline
		\textbf{Wintersemester 2017/18 (05.12.2017)} & Tobias Ullerich & 
			Analyse BVG Datenbank 2/2 \newline \\
		\hline
		\textbf{Wintersemester 2017/18 (19.12.2017)} & Tobias Ullerich & 
			Update Dokumentenstruktur \newline Update Aufgabenstellung \newline \\
		\hline
		\textbf{Wintersemester 2017/18 (23.12.2017)} & Tobias Ullerich & 
			Dokumentation Rest API (Route, Journey) \newline \\
		\hline
		\textbf{Wintersemester 2017/18 (24.12.2017)} & Tobias Ullerich & 
			Dokumentation Rest API (Vehicle) \newline \\
		\hline
		\textbf{Wintersemester 2017/18 (27.12.2017)} & Tobias Ullerich & 
			Dokumentation Server Datenbank \newline \\
		\hline
		\textbf{Wintersemester 2017/18 (30.12.2017)} & Tobias Ullerich & 
			Dokumentation Funktionsweise \newline \\
		\hline
 	\end{tabular}
 	\caption{Dokumentengeschichte}
  \end{table}

\chapter{Problemstellung}
\section{Ist-Zustand}
\section{Ziel}

\chapter{Aufgabenstellung}
Das zu entwickelnde System ist eine Komponente, die einem Busfahrer Informationen �ber die Busse einer Buslinie zur Verf�gung stellen soll. Dabei wird die Komponente den zeitlichen Abstand zum vorherigen und nachfolgenden Bus einer Linie visualisieren. Der eigene Bus wird durch eine Fahrtennummer im Vorfeld festgelegt. Die Visualisierung geschieht durch eine Android Applikation. F�r die Echtzeitdaten der Busse steht eine interne Datenschnittstelle der Berliner Verkehrsbetriebe (BVG) zur Verf�gung.
\\
Zu entwickelnde Komponenten:
\begin{itemize}
	\item{Schnittstelle zur betriebsinternen Schnittstelle der BVG} 
	\item{Persistierung der Daten in einer Datenbank} 
	\item{Aufbereitung der gesammelten Daten durch einen zu entwickelnden Algorithmus}
	\item{Visualisierung durch eine Android App} 
\end{itemize}

\chapter{Archtiketur}
\section{�berblick}

\section{Schnittstellendefinition}

\section{genutzte Komponenten}

\chapter{Nutzung}
\section{Code}
\section{Deployment / Runtime}


\chapter{Vorschl�ge / Ausblick}

\end{document}
