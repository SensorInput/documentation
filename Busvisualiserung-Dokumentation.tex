\documentclass[11pt,oneside,a4paper]{scrreprt}
\usepackage[a4paper, top=20mm, bottom=26mm, left=35mm, right=25mm]{geometry}

\usepackage[ngerman]{babel}
\usepackage[latin1]{inputenc}
\usepackage[T1]{fontenc}
\usepackage[babel,german=quotes]{csquotes}
\usepackage{amssymb}
\usepackage{amsmath}
\usepackage{graphicx}
\usepackage{wrapfig}
\usepackage{color}
\usepackage{palatino}
\usepackage{float}
\usepackage{subfigure} 
\usepackage{tikz}
\usepackage{acronym}

\usepackage[hyphens]{url}
\usepackage{xcolor}
\usepackage{listings}

\def\code#1{\texttt{#1}}

%Listing
\lstdefinelanguage{JavaScript}{
  keywords={typeof, new, true, false, catch, function, return, null, catch, switch, var, if, in, while, do, else, case, break},
  keywordstyle=\color{blue}\bfseries,
  ndkeywords={class, export, boolean, throw, implements, import, this},
  ndkeywordstyle=\color{darkgray}\bfseries,
  identifierstyle=\color{black},
  sensitive=false,
  comment=[l]{//},
  morecomment=[s]{/*}{*/},
  commentstyle=\color{purple}\ttfamily,
  stringstyle=\color{red}\ttfamily,
  morestring=[b]',
  morestring=[b]"
}

\definecolor{lightgray}{rgb}{0.98,0.98,0.98}
\definecolor{framegray}{rgb}{0.9,0.9,0.9}
\lstset{numbers=left,
	aboveskip=1.5em,
	basicstyle=\ttfamily,
	numberstyle=\footnotesize\ttfamily\color{darkgray},
	extendedchars=true,
	showspaces=false,
	showtabs=true,
	tabsize=3,
	breakautoindent=true,
	breakindent=2em,
	breaklines=true,
	frame=lbtr,
	xleftmargin=8mm,
	backgroundcolor=\color{lightgray},
	rulecolor=\color{framegray},
	%prebreak=\raisebox{-0.8ex}{\Righttorque},
	keywordstyle=\color[rgb]{0,0,1},
	commentstyle=\color[rgb]{0.133,0.545,0.133},
	stringstyle=\color[rgb]{0.627,0.126,0.941},
	language=JavaScript,
	moredelim=**[is][\color{gray}]{�}{�}
}

\begin{document}

\pagenumbering{gobble}

% Deckplatt
\begin{titlepage}
\centering
\includegraphics{res/logo}\par\vspace{2cm}
{\scshape\LARGE Hochschule f�r Technik und Wirtschaft \par}
\vspace{1cm}
{\scshape\Large Projektarrbeit\par}
\vspace{3cm}
{\huge\bfseries Visualisierung  und Auswertung von Positionsdaten der Omnibusse der BVG\par}
\vspace{1.5cm}
{\Large Technik mobiler Systeme\par}
{\Large Ausgew�hlte Kapitel mobiler Anwendungen\par}
\vspace{2cm}
{\Large\itshape Pascal Dettmers (551733)\par}
{\Large\itshape Stefan Neuberger (553849)\par}
{\Large\itshape Tobias Ullerich (553746)\par}
\vfill

{\large \today\par}
\end{titlepage}

% Abbildungs- und Tabellenverzeichnis
\tableofcontents

\listoffigures
\begingroup
\let\clearpage\relax
\listoftables
\endgroup

\pagenumbering{Roman}

% Abk�rzungsverzeichnis
\chapter*{Abk�rzungsverzeichnis}
\begin{acronym}[OSM]
	\acro{RBL}{rechnerbasiertes Leitsystem}
	\acro{OSM}{Open Street Map}
\end{acronym}

\newpage
\pagenumbering{arabic}

% TODO
% Def: Course
% Def: Route
% Def: Line

% Inhalt
\chapter{Dokumentengeschichte}
\begin{table}[h]
	\begin{tabular}{|l|l|p{5,5cm}|}
            	\hline
            	Zeitraum & TPL/Autor(en) & �nderungen \\
            	\hline
		\textbf{Wintersemester 2017/18 (09.11.2017)} & Tobias Ullerich & 
			Initiale Dokumentenstruktur \newline Entwurf Aufgabenstellung \newline \\
         	\hline
		\textbf{Wintersemester 2017/18 (01.12.2017)} & Tobias Ullerich & 
			Analyse BVG Datenbank 1/2 \newline \\
         	\hline
		\textbf{Wintersemester 2017/18 (05.12.2017)} & Tobias Ullerich & 
			Analyse BVG Datenbank 2/2 \newline \\
		\hline
		\textbf{Wintersemester 2017/18 (19.12.2017)} & Tobias Ullerich & 
			Update Dokumentenstruktur \newline Update Aufgabenstellung \newline \\
		\hline
		\textbf{Wintersemester 2017/18 (23.12.2017)} & Tobias Ullerich & 
			Dokumentation Rest API (Route, Journey) \newline \\
		\hline
		\textbf{Wintersemester 2017/18 (24.12.2017)} & Tobias Ullerich & 
			Dokumentation Rest API (Vehicle) \newline \\
		\hline
		\textbf{Wintersemester 2017/18 (27.12.2017)} & Tobias Ullerich & 
			Dokumentation Server Datenbank \newline \\
		\hline
		\textbf{Wintersemester 2017/18 (30.12.2017)} & Tobias Ullerich & 
			Dokumentation Funktionsweise \newline \\
		\hline
 	\end{tabular}
 	\caption{Dokumentengeschichte}
  \end{table}

\chapter{Problemstellung}
Problemstellung Busbunching, BVG, Ist-Zustand, Ziel

\chapter{Aufgabenstellung}
Das zu entwickelnde System ist eine Komponente, die einem Busfahrer Informationen �ber die Busse einer Buslinie zur Verf�gung stellen soll. Dabei soll das System nicht nur auf Busse beschr�nkt sein. Denkbar sind hier auch andere Arten von �ffentlichen Transportmitteln. Dabei wird die Komponente den zeitlichen Abstand zum vorherigen und nachfolgenden Bus einer Linie visualisieren. Die Visualisierung geschieht durch eine Android Applikation.

\chapter{Archtiketur}
\section{�berblick}
\subsection{Analyse Datenbank BVG}
Die BVG nutzt f�r die Persistierung der Daten, inklusive der Prozessdaten, ein Datenbanksystem der Firma Oracle. Es werden bei der BVG zwischen zwei verschiedenen Systemen unterschieden. Zum Einen gibt es die sogenannte SC05 Schnittstelle. Diese enth�lt Prozessdaten der aktuellen Betriebslage. Dazu z�hlen unter anderem Positionen von Bussen und deren Versp�tung (vgl.: "`Die Prozessdatenschnittstelle (SC05) spiegelt die aktuelle Situation im RBL wider."'). % TODO
Zum Anderen gibt es die SC51 Datenbank, entwickelt von der Firma Alcatel. Diese Schnittstelle enth�lt unterschiedlichste Daten f�r die Durchf�hrung des �ffentlichen Nahverkehrs der BVG. Darunter fallen Informationen zu Linien (Bus und Bahn), Informationen �ber deren Routen mittels geografischer Koordinaten und vieles mehr. 
F�r die Analyse dieser relationalen Datenbanken waren jeweils Dokumentationen und ein Dump der Datenbank zur Verf�gung.
 
Der erste Schritt bei der Analyse bestand darin, die Dumps der Oracle Datenbanken zu importieren, um anschlie�end Zugriff auf die Tabellen und deren Daten zu erlangen. F�r den Import fiel die Entscheidung f�r das Tool "`OraDump to MySQL"'. % TODO https://www.convert-in.com/ord2sql.htm
Mit diesem Tool ist es m�glich ein Oracle Datenbank Dump in eine MySQL Datenbank zu importieren. Vorteil dieser Methode ist, das auf bestehende Kenntnisse im Umgang mit MySQL zur�ckgegriffen werden kann. Im folgenden wurde mittels der Schnittstellendokumentation die Struktur der Datenbank analysiert. Im Fokus dieser Analyse stehen die Routen Information aus der SC51 und die Positionsdaten der Fahrzeuge aus der SC05 Schnittstelle. Bei der Analyse haben sich folgende Datenbanktabellen als wertvoll gezeigt. 
 
Die Tabelle \code{CM\_VEHICLE\_POSITION} aus der SC05 Datenbank enth�lt Informationen zu der aktuellen geografischen Position mittels Latitude und Longitude, der Abweichung vom Sollfahrplan in Sekunden, sowie eine Zuordnung zu einer Route. Um einen Omnibus auf einer Route einzuordnen, gibt es eine endliche Menge von geografischen Punkten. Zu all diesen Punkten ist ein zeitlicher und �rtlicher Abstand bekannt (siehe SC51). Zu jedem Fahrzeug ist der letzte passierte Punkt der Route referenziert (\code{LAST\_POR\_ORDER}). Der prozentuale Abstand zum Folgepunkt auf einer Route ist ebenfalls in der Relation durch die Spalte \code{REL\_LNK\_DISTANCE} gegeben.

F�r die Zuordnung der Fahrzeuge aus der Tabelle \code{CM\_VEHICLE\_POSITION} zu einer Route und einem Kurs gibt es in der SC05 Datenbank zwei Tabellen. Zum Einen hat die Tabelle \code{CM\_ACCT\_COURSES} die Aufgabe, ein Fahrzeug einem Kurs zuzuordnen. Zum Anderen wird durch \code{CM\_ACCT\_JOURNEY} ein Bus einer Route zugeordnet. Somit k�nnen die \code{POINTS\_ON\_ROUTE} einem Fahrzeug zugeordnet werden.

Die Datenbank SC51 beinhaltet Tabellen f�r die Linien (\code{LINES}). Eine Linie ist im Kontext der BVG zum Beispiel die konkrete Buslinie X11. Jede Linie besteht aus mehren Fahrten, hier \code{COURSES\_ON\_JOURNEY} genannt. Zu einem Kurs geh�ren Informationen wie Startzeit, Endzeit und eine Kursnummer, die nur im Kontext einer Linie eindeutig ist.

Die geografischen Informationen zum Routenverlauf werden in den Tabellen \code{ROUTE}, \code{POINTS\_ON\_ROUTE} und \code{NETWORK\_POINTS} verwaltet. Zu einer Buslinie k�nnen verschiedene Routen geh�ren. Diese Routen sind in der Tabelle \code{ROUTE} zu finden. Daf�r enth�lt auch diese Relation zus�tzlich ein Feld \code{Description} (Beispieldaten: Falkensee, Bahnhof->S+U Rathaus Spandau). Die Tabelle \code{POINTS\_ON\_ROUTE} koordiniert die Punkte einer Route, indem jeder Punkt einen Laufnummer hat (\code{POR\_ORDER}). Um den zeitlichen und �rtlichen Abstand zwischen zwei Punkten zu ermitteln, wird die Relation \code{LINKS} verwendet. Die Tabelle \code{NETWORK\_POINTS} enth�lt abschlie�end die eigentlichen geografischen Punkte in der Form Latitude und Longitude.

In der Abbildung \ref{img:db-bvg} sind die Zusammenh�nge der einzelnen Datenbanktabelle von SC05 und SC51 zu sehen. Dabei handelt es sich lediglich um einen Auszug der relevanten Daten f�r das zu entwickelnde System.

\begin{figure}[H]
	\centering
	\includegraphics[width=15cm]{res/BVG-DB.png}
	\caption{Datenbank Schema aus SC05 und SC51}
	\label{img:db-bvg}
\end{figure}

\subsection{OSM}
Wie kommt man an die Routen Infos zu den Bus-/Tram Linien. GeoJson.

\section{Komponenten}
Bestandteile des Systems beschreiben (Server, App, ...)

\section{Rest API Schnittstellendefinition}
Der Server f�r die Datenverwaltung basiert auf einer Rest API. Im folgenden Abschnitt werden alle m�glichen Request zu Resourcen allgemein und mithilfe eines Beispiels echter Daten beschrieben. W�hrend der Projektarbeit besteht die M�glichkeit, die API �ber den Server \url{http://bus.f4.htw-berlin.de:4545} aus dem HTW-Berlin Netz zu nutzen.
% ##########    GET  /api/v1/route/<ref>    ###########
\subsection{Request f�r Routen einer Linie}
Anfrage aller Routen einer Linie (Beispiel Buslinie). Eine Linie hat mehrere Routen.

\subsubsection{HTTP-Method}
GET

\subsubsection{Resource}
/api/v1/route/\textless ref\textgreater

\subsubsection{Parameters}
\begin{itemize}
	\item ref: Liniennummer
\end{itemize}

\subsubsection{Response}
\begin{lstlisting}
[ {
  "id" : <id>,
  "osmId" : "relation/<osmId>",
  "ref" : "<line number>",
  "name" : "<description of the route>",
  "type" : "<transport type>",
  "network" : "<full name of network>",
  "operator" : "<full name of operator>",
  "from" : "<name of start station>",
  "to" : "<name of end station>",
  "routeType" : "<MULTILINE | LINE>"
} ]
\end{lstlisting}

\subsubsection{HTTP status codes}
\begin{itemize}
	\item 200: Request erfolgreich
	\item 404: Keine Routen gefunden
\end{itemize}

\subsubsection{Beispiel Request}
GET \url{http://domain.com/api/v1/route/M11}

\subsubsection{Beispiel Response}
\begin{lstlisting}
[ {
  "id" : 217,
  "osmId" : "relation/2088816",
  "ref" : "M11",
  "name" : "Buslinie M11: S Sch�neweide => U Dahlem Dorf",
  "type" : "bus",
  "network" : "Verkehrsverbund Berlin-Brandenburg",
  "operator" : "Berliner Verkehrsbetriebe",
  "from" : "S Sch�neweide",
  "to" : "U Dahlem Dorf",
  "routeType" : "MULTILINE"
}, {
  "id" : 218,
  "osmId" : "relation/2088817",
  "ref" : "M11",
  "name" : "Buslinie M11: U Dahlem Dorf => S Sch�neweide",
  "type" : "bus",
  "network" : "Verkehrsverbund Berlin-Brandenburg",
  "operator" : "Berliner Verkehrsbetriebe",
  "from" : "U Dahlem Dorf",
  "to" : "S Sch�neweide",
  "routeType" : "MULTILINE"
} ]
\end{lstlisting}


% ##########    GET  /api/v1/route/geo/<ref>    ###########
\subsection{Request f�r GeoJson einer Route mit ID}
Anfrage einer Route per Id. Das Format der Response ist GeoJson Format.

\subsubsection{HTTP-Method}
GET

\subsubsection{Resource}
/api/v1/route/geo/\textless id\textgreater

\subsubsection{Parameters}
\begin{itemize}
	\item id: Route ID
\end{itemize}

\subsubsection{Response}
\begin{lstlisting}
{
  "type": "Feature",
  "properties": {
    "ref": "<id>",
    "name": "<description of the route>",
    "@id": "relation/<osmId>"
    "from": "<name of start station>",
    "to": "<name of end station>",
    "type": "<transport type>",
    "operator": "<full name of operator>",
    "network": "<full name of network>"
  },
  "geometry": {
    "type": "<MultiLineString | LineString>",
    "coordinates":[]
  }
}
\end{lstlisting}

\subsubsection{HTTP status codes}
\begin{itemize}
	\item 200: Request erfolgreich
	\item 400: Ung�ltiger Parameter
	\item 404: Keine Route gefunden
\end{itemize}

\subsubsection{Beispiel Request}
GET \url{http://domain.com/api/v1/route/geo/67}

\subsubsection{Beispiel Response}
\begin{lstlisting}
{
  "type": "Feature",
  "properties": {
    "ref": "67",
    "name": "Stra�enbahnlinie 67: Krankenhaus K�penick => S Sch�neweide",
    "from": "Krankenhaus K�penick",
    "@id": "relation/2084473",
    "to": "S Sch�neweide",
    "type": "tram",
    "operator": "Berliner Verkehrsbetriebe",
    "network": "Verkehrsverbund Berlin-Brandenburg"
  },
  "geometry": {
    "type": "MultiLineString",
    "coordinates": [
      [
        [
          13.5939995,
          52.4385062
        ],
        [
          13.5939794,
          52.438533
        ], ....
    ]
  }
}
\end{lstlisting}


% ##########    POST  /api/v1/route    ###########
\subsection{Hinzuf�gen einer Route}
F�gt eine Route (oder mehrere) hinzu. Der Payload muss im GeoJson Format sein.

\subsubsection{HTTP-Method}
POST

\subsubsection{Resource}
/api/v1/route

\subsubsection{Payload}
\begin{lstlisting}
{
  "type": "FeatureCollection",
  "features": [
    {
      "type": "Feature",
      "properties": {
        "@id": "relation/<osmId>",
        "name": "<description of the route>",
        "network": "<full name of network>",
        "operator": "<full name of operator>",
        "from": "<name of start station>",
        "to": "<name of end station>",
        "ref": "<id>",
        "route": "<transport type>",
        "type": "route",
      },
      "geometry": {
        "type": "<MultiLineString | LineString>",
        "coordinates": [
        ]
     }
  }
}
\end{lstlisting}

\subsubsection{HTTP status codes}
\begin{itemize}
	\item 201: Resource erstellt
	\item 500: Bad payload
\end{itemize}

\subsubsection{Beispiel Request}
POST \url{http://domain.com/api/v1/route}

\begin{lstlisting}
{
  "type": "Feature",
  "properties": {
    "ref": "67",
    "name": "Stra�enbahnlinie 67: Krankenhaus K�penick => S Sch�neweide",
    "from": "Krankenhaus K�penick",
    "@id": "relation/2084473",
    "to": "S Sch�neweide",
    "type": "tram",
    "operator": "Berliner Verkehrsbetriebe",
    "network": "Verkehrsverbund Berlin-Brandenburg"
  },
  "geometry": {
    "type": "MultiLineString",
    "coordinates": [
       [
         [
         13.5939995,
         52.4385062
       ],
       [
         13.5939794,
         52.438533
      ], ....
    ]
  }
}
\end{lstlisting}


% ##########    GET  /api/v1/journey    ###########
\subsection{Request f�r eine Journey}
Anfrage f�r Messwerte einer Fahrt (Journey) auf einer Route. Die Antwort enth�lt die bereits gegl�tteten Koordinaten.

\subsubsection{HTTP-Method}
GET

\subsubsection{Resource}
/api/v1/journey/\textless id\textgreater

\subsubsection{Parameters}
\begin{itemize}
	\item id: Journey ID
\end{itemize}

\subsubsection{Response}
\begin{lstlisting}
{
  "id": <journeyId>,
  "routeId": <routeId>,
  "startTime": <timeStamp>,
  "endTime": <timeStamp>,
  "points": [
    {
      "id": <id>,
      "journeyId": <journeyId>,
      "time": <timeStamp>,
      "lngLat": {
        "lng": <longitude>,
        "lat": <latidute>
      }
    }, ...
  ]
}
\end{lstlisting}

\subsubsection{HTTP status codes}
\begin{itemize}
	\item 200: Anfrage erfolgreich
	\item 400: Ung�ltiger Parameter
	\item 404: Keine Journey gefunden
\end{itemize}

\subsubsection{Beispiel Request}
GET \url{http://domain.com/api/v1/journey/2}

\subsubsection{Beispiel Response}
\begin{lstlisting}
{
  "id": 2,
  "routeId": 67,
  "startTime": 1513596120000,
  "endTime": 1513597500000,
  "points": [
    {
      "id": 93,
      "journeyId": 2,
      "time": 1513596185100,
      "lngLat": {
        "lng": 13.5916396,
        "lat": 52.4390415
      }
    }, ...
  ]
}
\end{lstlisting}


% ##########    POST  /api/v1/journey    ###########
\subsection{Hinzuf�gen einer neuen Journey}
Erstellt eine neue Journey.

\subsubsection{HTTP-Method}
POST

\subsubsection{Resource}
/api/v1/journery

\subsubsection{Payload}
\begin{lstlisting}
{
  "routeId" : <routeId>,
  "startTime": <timestamp>,
  "endTime": <timestamp>
}
\end{lstlisting}


\subsubsection{Response}
\begin{lstlisting}
<id>
\end{lstlisting}

\subsubsection{HTTP status codes}
\begin{itemize}
	\item 201: Resource erstellt
	\item 400: Bad payload
\end{itemize}

\subsubsection{Beispiel Request}
POST \url{http://domain.com/api/v1/journey}

\begin{lstlisting}
{
  "routeId" : 67,
  "startTime": 1513596120000,
  "endTime": 1513597500000
}
\end{lstlisting}

\subsubsection{Beispiel Response}
\begin{lstlisting}
7
\end{lstlisting}


% ##########    POST  /api/v1/journey/position    ###########
\subsection{Hinzuf�gen einer Messposition f�r die Zeitmessung}
F�gt einen neuen Messpunkt f�r eine Route einer Journey hinzu. Eine Journey beschreibt eine konkrete Fahrt auf einer konkreten Route.

\subsubsection{HTTP-Method}
POST

\subsubsection{Resource}
/api/v1/journey/position

\subsubsection{Payload}
\begin{lstlisting}
{
  "journeyId" : <journeyId>,
  "time": <timestamp>,
  "lngLat" : {
    "lng": <longitude>,
    "lat": <latitude>
  }
}
\end{lstlisting}


\subsubsection{Response}
\begin{lstlisting}
<id>
\end{lstlisting}

\subsubsection{HTTP status codes}
\begin{itemize}
	\item 201: Resource erstellt
	\item 400: Bad payload
\end{itemize}

\subsubsection{Beispiel Request}
POST \url{http://domain.com/api/v1/journey/position}

\begin{lstlisting}
{
  "time":1467888902,
  "journeyId": 3,
  "lngLat": {
    "lat":13.43546,
    "lng":52.32332	
  }
}
\end{lstlisting}

\subsubsection{Beispiel Response}
\begin{lstlisting}
251
\end{lstlisting}


% ##########    GET  /api/v1/vehicle    ###########
\subsection{Request f�r ein Fahrzeug}
Anfrage f�r ein Fahrzeug. Die Antwort enth�lt Geo Daten und Routeninformationen.

\subsubsection{HTTP-Method}
GET

\subsubsection{Resource}
/api/v1/vehicle/\textless id\textgreater

\subsubsection{Parameters}
\begin{itemize}
	\item id: Unique Id des Fahrzeuges (Devices)
\end{itemize}

\subsubsection{Response}
\begin{lstlisting}
{
  "id": <id>,
  "ref": "<uniqueId>",
  "routeId": <routeId>,
  "time": <timeStamp>,
  "position": {
    "lng": <longitude>,
    "lat": <latitude>
  },
  "pastedDistance" : <meter>
}
\end{lstlisting}

\subsubsection{HTTP status codes}
\begin{itemize}
	\item 200: Anfrage erfolgreich
	\item 404: Kein Fahrzeug gefunden
\end{itemize}

\subsubsection{Beispiel Request}
GET \url{http://domain.com/api/v1/journey/2}

\subsubsection{Beispiel Response}
\begin{lstlisting}
{
  "id": 1,
  "ref": "636c81cc2361acd7",
  "routeId": 67,
  "time": 1513596185100,
  "position": {
    "lng": 52.3453,
    "lat": 13.53234
  },
  "pastedDistance" : 5592.54969445254
}
\end{lstlisting}


% ##########    POST  /api/v1/vehicle    ###########
\subsection{Hinzuf�gen eines Fahrzeuges}
F�gt ein neues Fahrzeug der Datenbank hinzu.

\subsubsection{HTTP-Method}
POST

\subsubsection{Resource}
/api/v1/vehicle

\subsubsection{Payload}
\begin{lstlisting}
{
  "ref": "<uniqueId>",
  "routeId": <routeId>,
  "time": <timeStamp>,
  "position": {
    "lng": <longitude>,
    "lat": <latitude>
  }
}
\end{lstlisting}


\subsubsection{Response}
\begin{lstlisting}
<id>
\end{lstlisting}

\subsubsection{HTTP status codes}
\begin{itemize}
	\item 201: Resource erstellt
	\item 400: Bad payload
\end{itemize}

\subsubsection{Beispiel Request}
POST \url{http://domain.com/api/v1/vehicle}

\begin{lstlisting}
{
  "ref": "636c81cc2361acd7",
  "routeId": 67,
  "time": 1513596185100,
  "position": {
    "lng": 52.3453,
    "lat": 13.53234
  }
}
\end{lstlisting}

\subsubsection{Beispiel Response}
\begin{lstlisting}
2
\end{lstlisting}


% ##########    PUT  /api/v1/vehicle    ###########
\subsection{Update eines Fahrzeuges} \label{sub:vehicle:put}
Aktualisiert die Daten ein neues Fahrzeuges int der Datenbank hinzu.

\subsubsection{HTTP-Method}
PUT

\subsubsection{Resource}
/api/v1/vehicle

\subsubsection{Payload}
\begin{lstlisting}
{
  "ref": "<uniqueId>",
  "routeId": <routeId>,
  "time": <timeStamp>,
  "position": {
    "lng": <longitude>,
    "lat": <latitude>
  }
}
\end{lstlisting}

\subsubsection{HTTP status codes}
\begin{itemize}
	\item 201: Resource erstellt
	\item 400: Bad payload
\end{itemize}

\subsubsection{Beispiel Request}
PUT \url{http://domain.com/api/v1/vehicle}

\begin{lstlisting}
{
  "ref": "636c81cc2361acd7",
  "routeId": 67,
  "time": 1513596185100,
  "position": {
    "lng": 52.3453,
    "lat": 13.53234
  }
}
\end{lstlisting}


% ##########    GET  /api/v1/vehicle/id/list    ###########
\subsection{Request f�r ein Fahrzeug} \label{sub:vehicle:list}
Anfrage aller Fahrzeuge, die mit einem Fahrzeug in Verbindung stehen. Das bedeutet, die Antwort enth�lt alle Fahrzeuge auf der Route des angefragten Fahrzeuges.

\subsubsection{HTTP-Method}
GET

\subsubsection{Resource}
/api/v1/vehicle/\textless id\textgreater/list

\subsubsection{Parameters}
\begin{itemize}
	\item id: Unique Id des Fahrzeuges (Devices)
\end{itemize}

\subsubsection{Response}
\begin{lstlisting}
[ {
  "ref" : "<devideId>",
  "geoLngLat" : {
    "lng" : <longitude>,
    "lat" : <latitude>
  },
  "relativeDistance" : <distanceMetersToRequestedVehicle>,
  "relativeTimeDistance": <distanceMilliSecondsToRequestedVehicle>
} ]
\end{lstlisting}

\subsubsection{HTTP status codes}
\begin{itemize}
	\item 200: Anfrage erfolgreich
	\item 404: Kein Fahrzeug gefunden
\end{itemize}

\subsubsection{Beispiel Request}
GET \url{http://domain.com/api/v1/vehicle/636c81cc2361acd7/list}

\subsubsection{Beispiel Response}
\begin{lstlisting}
[ {
  "ref" : "636c81cc2361acd7",
  "geoLngLat" : {
    "lng" : 13.5395005,
    "lat" : 52.4575977
  },
  "relativeDistance" : 0.0
}, {
  "ref" : "4dcghc4zzc6cghcf",
  "geoLngLat" : {
    "lng" : 13.5716218,
    "lat" : 52.4511964
  },
  "relativeDistance" : 2504.827656693912,
  "relativeTimeDistance": 411008
} ]
\end{lstlisting}



\section{Datenbankschema Server}
F�r die Persistierung der Daten des Servers wurde eine relationale Datenbank gew�hlt. Speziell f�r die Implementierung wird eine MySQL Datenbank gew�hlt. Um auf die Datenbank aus Java zuzugreifen, wird der Java Database Connector (JDBC) verwendet.

In der Datenbank werden Daten der Route und deren Geometrien aus OSM gespeichert. Daf�r gibt es die Tabellen \code{Route}, \code{MultiLineString} und \code{LineString}. Die Relation \code{Route} enth�lt Informationen �ber die Liniennummern (\code{rel}), Start- und Zielhaltestelle (\code{from}, \code{to}). Zudem sind Informationen �ber das Verkehrsunternehmen (\code{operator}, Beispiel: BVG) und den Verkehrsverbund (\code{network}, Beispiel: VBB) vorhanden. Die Eigenschaft \code{type} entscheidet �ber den Typ der Geometrie einer Route. Die eigentliche Information �ber die Geometrie befindet sich als String in der Tabelle \code{MultiLineString} oder \code{LineString}. Ein solcher MultiLineString kann beispielsweise wie folgt aussehen: 
\begin{lstlisting}
MultiLineString((13.6920278 52.4516269, 13.6923877 52.4515733, 13.6926666 52.4515393, ...))
\end{lstlisting}

Um Aussagen �ber die Geschwindigkeit von Fahrzeugen auf einer Route treffen zu k�nnen, wurden im Vorfeld Testfahrten gemacht. Die gesammelten Daten w�hrend dieser Testfahrten sind in den Tabellen \code{Journey} und \code{MeasurePoint} persistiert. Die Tabelle \code{Journey} beschreibt eine spezielle Fahrt zu einer Uhrzeit auf einer Route. Die Messwerte, also die GPS Position eines Fahrzeugs und Zeitstempel, werden zu einer Fahrt in der Relation \code{MeasurePoint} gespeichert. Bei den GPS Daten handelt es sich um die ungegl�tteten Daten, das hei{\ss}t also, dass die GPS Daten nicht zwingend auf der Route liegen.

In der Tabelle \code{Vehicle} werden die aktuellen Informationen zu den Fahrzeugen gespeichert. Daf�r enth�lt die Tabelle Informationen zu GPS Position (gegl�ttet), einen Zeitstempel und die Route eines Fahrzeuges. Jedes Fahrzeug wird �ber einen Unique Identifier identifiziert (\code{ref}).

Um diese Fahrzeugdaten historisch zu persistieren, gibt es zus�tzlich  zu \code{Vehicle}�die Relation \code{VehicleHistory}. In dieser Tabelle existiert die gleiche Struktur wie \code{Vehicle}. Zus�tzlich gibt es noch einen k�nstlichen Prim�rschl�ssel, um jede Zeile eindeutig zu identifizieren. Somit ist es m�glich zu jedem Zeitpunkt herauszufinden, auf welcher Route und an welcher Stelle ein Fahrzeug zu einem Zeitpunkt war.

Die Abbildung \ref{img:db-server} visualisiert die erstellte Datenbankstruktur mit den Relationen und deren Beziehungen f�r den Server.

\begin{figure}[H]
	\centering
	\includegraphics[width=15cm]{res/Server-DB.png}
	\caption{Datenbank Schema Server}
	\label{img:db-server}
\end{figure}


\chapter{Nutzung}
\section{Code}
\subsection{Programmiersprache}

\subsection{Bibliotheken}

\section{Funktionsweise}
Wie funktioniert die Berechnung.

\section{Deployment / Runtime}


\chapter{Vorschl�ge / Ausblick}

\chapter{Literaturverzeichnis}

\end{document}
