\section{Rest API Schnittstellendefinition}
Der Server f�r die Datenverwaltung basiert auf einer Rest API. Im folgenden Abschnitt werden alle m�glichen Request zu Resourcen allgemein und mithilfe eines Beispiels echter Daten beschrieben. W�hrend der Projektarbeit besteht die M�glichkeit, die API �ber den Server \url{http://bus.f4.htw-berlin.de:4545} aus dem HTW-Berlin Netz zu nutzen.
% ##########    GET  /api/v1/route/<ref>    ###########
\subsection{Request f�r Routen einer Linie}
Anfrage aller Routen einer Linie (Beispiel Buslinie). Eine Linie hat mehrere Routen.

\subsubsection{HTTP-Method}
GET

\subsubsection{Resource}
/api/v1/route/\textless ref\textgreater

\subsubsection{Parameters}
\begin{itemize}
	\item ref: Liniennummer
\end{itemize}

\subsubsection{Response}
\begin{lstlisting}
[ {
  "id" : <id>,
  "osmId" : "relation/<osmId>",
  "ref" : "<line number>",
  "name" : "<description of the route>",
  "type" : "<transport type>",
  "network" : "<full name of network>",
  "operator" : "<full name of operator>",
  "from" : "<name of start station>",
  "to" : "<name of end station>",
  "routeType" : "<MULTILINE | LINE>"
} ]
\end{lstlisting}

\subsubsection{HTTP status codes}
\begin{itemize}
	\item 200: Request erfolgreich
	\item 404: Keine Routen gefunden
\end{itemize}

\subsubsection{Beispiel Request}
GET \url{http://domain.com/api/v1/route/M11}

\subsubsection{Beispiel Response}
\begin{lstlisting}
[ {
  "id" : 217,
  "osmId" : "relation/2088816",
  "ref" : "M11",
  "name" : "Buslinie M11: S Sch�neweide => U Dahlem Dorf",
  "type" : "bus",
  "network" : "Verkehrsverbund Berlin-Brandenburg",
  "operator" : "Berliner Verkehrsbetriebe",
  "from" : "S Sch�neweide",
  "to" : "U Dahlem Dorf",
  "routeType" : "MULTILINE"
}, {
  "id" : 218,
  "osmId" : "relation/2088817",
  "ref" : "M11",
  "name" : "Buslinie M11: U Dahlem Dorf => S Sch�neweide",
  "type" : "bus",
  "network" : "Verkehrsverbund Berlin-Brandenburg",
  "operator" : "Berliner Verkehrsbetriebe",
  "from" : "U Dahlem Dorf",
  "to" : "S Sch�neweide",
  "routeType" : "MULTILINE"
} ]
\end{lstlisting}


% ##########    GET  /api/v1/route/geo/<ref>    ###########
\subsection{Request f�r GeoJson einer Route mit ID}
Anfrage einer Route per Id. Das Format der Response ist GeoJson Format.

\subsubsection{HTTP-Method}
GET

\subsubsection{Resource}
/api/v1/route/geo/\textless id\textgreater

\subsubsection{Parameters}
\begin{itemize}
	\item id: Route ID
\end{itemize}

\subsubsection{Response}
\begin{lstlisting}
{
  "type": "Feature",
  "properties": {
    "ref": "<id>",
    "name": "<description of the route>",
    "@id": "relation/<osmId>"
    "from": "<name of start station>",
    "to": "<name of end station>",
    "type": "<transport type>",
    "operator": "<full name of operator>",
    "network": "<full name of network>"
  },
  "geometry": {
    "type": "<MultiLineString | LineString>",
    "coordinates":[]
  }
}
\end{lstlisting}

\subsubsection{HTTP status codes}
\begin{itemize}
	\item 200: Request erfolgreich
	\item 404: Keine Route gefunden
\end{itemize}

\subsubsection{Beispiel Request}
GET \url{http://domain.com/api/v1/route/geo/67}

\subsubsection{Beispiel Response}
\begin{lstlisting}
{
  "type": "Feature",
  "properties": {
    "ref": "67",
    "name": "Stra�enbahnlinie 67: Krankenhaus K�penick => S Sch�neweide",
    "from": "Krankenhaus K�penick",
    "@id": "relation/2084473",
    "to": "S Sch�neweide",
    "type": "tram",
    "operator": "Berliner Verkehrsbetriebe",
    "network": "Verkehrsverbund Berlin-Brandenburg"
  },
  "geometry": {
    "type": "MultiLineString",
    "coordinates": [
      [
        [
          13.5939995,
          52.4385062
        ],
        [
          13.5939794,
          52.438533
        ], ....
    ]
  }
}
\end{lstlisting}


% ##########    POST  /api/v1/route    ###########
\subsection{Hinzuf�gen einer Route}
F�gt eine Route (oder mehrere) hinzu. Der Payload muss im GeoJson Format sein.

\subsubsection{HTTP-Method}
POST

\subsubsection{Resource}
/api/v1/route

\subsubsection{Payload}
\begin{lstlisting}
{
  "type": "FeatureCollection",
  "features": [
    {
      "type": "Feature",
      "properties": {
        "@id": "relation/<osmId>",
        "name": "<description of the route>",
        "network": "<full name of network>",
        "operator": "<full name of operator>",
        "from": "<name of start station>",
        "to": "<name of end station>",
        "ref": "<id>",
        "route": "<transport type>",
        "type": "route",
      },
      "geometry": {
        "type": "<MultiLineString | LineString>",
        "coordinates": [
        ]
     }
  }
}
\end{lstlisting}

\subsubsection{HTTP status codes}
\begin{itemize}
	\item 201: Resource erstellt
	\item 500: Bad payload
\end{itemize}

\subsubsection{Beispiel Request}
POST \url{http://domain.com/api/v1/route}

\begin{lstlisting}
{
  "type": "Feature",
  "properties": {
    "ref": "67",
    "name": "Stra�enbahnlinie 67: Krankenhaus K�penick => S Sch�neweide",
    "from": "Krankenhaus K�penick",
    "@id": "relation/2084473",
    "to": "S Sch�neweide",
    "type": "tram",
    "operator": "Berliner Verkehrsbetriebe",
    "network": "Verkehrsverbund Berlin-Brandenburg"
  },
  "geometry": {
    "type": "MultiLineString",
    "coordinates": [
       [
         [
         13.5939995,
         52.4385062
       ],
       [
         13.5939794,
         52.438533
      ], ....
    ]
  }
}
\end{lstlisting}


% ##########    GET  /api/v1/journey    ###########
\subsection{Request f�r eine Journey}
Anfrage f�r Messwerte einer Fahrt (Journey) auf einer Route. Die Antwort enth�lt die bereits gegl�tteten Koordinaten.

\subsubsection{HTTP-Method}
GET

\subsubsection{Resource}
/api/v1/journey/\textless id\textgreater

\subsubsection{Parameters}
\begin{itemize}
	\item id: Journey ID
\end{itemize}

\subsubsection{Response}
\begin{lstlisting}
{
  "id": <journeyId>,
  "routeId": <routeId>,
  "startTime": <timeStamp>,
  "endTime": <timeStamp>,
  "points": [
    {
      "id": <id>,
      "journeyId": <journeyId>,
      "time": <timeStamp>,
      "lngLat": {
        "lng": <longitude>,
        "lat": <latidute>
      }
    }, ...
  ]
}
\end{lstlisting}

\subsubsection{HTTP status codes}
\begin{itemize}
	\item 200: Anfrage erfolgreich
	\item 404: Keine Journey gefunden
\end{itemize}

\subsubsection{Beispiel Request}
GET \url{http://domain.com/api/v1/journey/2}

\subsubsection{Beispiel Response}
\begin{lstlisting}
{
  "id": 2,
  "routeId": 67,
  "startTime": 1513596120000,
  "endTime": 1513597500000,
  "points": [
    {
      "id": 93,
      "journeyId": 2,
      "time": 1513596185100,
      "lngLat": {
        "lng": 13.5916396,
        "lat": 52.4390415
      }
    }, ...
  ]
}
\end{lstlisting}


% ##########    POST  /api/v1/journey    ###########
\subsection{Hinzuf�gen einer neuen Journey}
Erstellt eine neue Journey.

\subsubsection{HTTP-Method}
POST

\subsubsection{Resource}
/api/v1/journery

\subsubsection{Payload}
\begin{lstlisting}
{
  "routeId" : <routeId>,
  "startTime": <timestamp>,
  "endTime": <timestamp>
}
\end{lstlisting}


\subsubsection{Response}
\begin{lstlisting}
<id>
\end{lstlisting}

\subsubsection{HTTP status codes}
\begin{itemize}
	\item 201: Resource erstellt
	\item 500: Bad payload
\end{itemize}

\subsubsection{Beispiel Request}
POST \url{http://domain.com/api/v1/journey}

\begin{lstlisting}
{
  "routeId" : 67,
  "startTime": 1513596120000,
  "endTime": 1513597500000
}
\end{lstlisting}

\subsubsection{Beispiel Response}
\begin{lstlisting}
7
\end{lstlisting}


% ##########    POST  /api/v1/journey/position    ###########
\subsection{Hinzuf�gen einer Messposition f�r die Zeitmessung}
F�gt einen neuen Messpunkt f�r eine Route einer Journey hinzu. Eine Journey beschreibt eine konkrete Fahrt auf einer konkreten Route.

\subsubsection{HTTP-Method}
POST

\subsubsection{Resource}
/api/v1/position

\subsubsection{Payload}
\begin{lstlisting}
{
  "journeyId" : <journeyId>,
  "time": <timestamp>,
  "lngLat" : {
    "lng": <longitude>,
    "lat": <latitude>
  }
}
\end{lstlisting}


\subsubsection{Response}
\begin{lstlisting}
<id>
\end{lstlisting}

\subsubsection{HTTP status codes}
\begin{itemize}
	\item 201: Resource erstellt
	\item 500: Bad payload
\end{itemize}

\subsubsection{Beispiel Request}
POST \url{http://domain.com/api/v1/position}

\begin{lstlisting}
{
  "time":1467888902,
  "journeyId": 3,
  "lngLat": {
    "lat":13.43546,
    "lng":52.32332	
  }
}
\end{lstlisting}

\subsubsection{Beispiel Response}
\begin{lstlisting}
251
\end{lstlisting}


% ##########    GET  /api/v1/vehicle    ###########
\subsection{Request f�r ein Fahrzeug}
Anfrage f�r ein Fahrzeug. Die Antwort enth�lt Geo Daten und Routeninformationen.

\subsubsection{HTTP-Method}
GET

\subsubsection{Resource}
/api/v1/vehicle/\textless id\textgreater

\subsubsection{Parameters}
\begin{itemize}
	\item id: Unique Id des Fahrzeuges (Devices)
\end{itemize}

\subsubsection{Response}
\begin{lstlisting}
{
  "id": <id>,
  "ref": "<uniqueId>",
  "routeId": <routeId>,
  "time": <timeStamp>,
  "position": {
    "lng": <longitude>,
    "lat": <latitude>
  }
}
\end{lstlisting}

\subsubsection{HTTP status codes}
\begin{itemize}
	\item 200: Anfrage erfolgreich
	\item 404: Kein Fahrzeug gefunden
\end{itemize}

\subsubsection{Beispiel Request}
GET \url{http://domain.com/api/v1/journey/2}

\subsubsection{Beispiel Response}
\begin{lstlisting}
{
  "id": 1,
  "ref": "636c81cc2361acd7",
  "routeId": 67,
  "time": 1513596185100,
  "position": {
    "lng": 52.3453,
    "lat": 13.53234
  }
}
\end{lstlisting}


% ##########    POST  /api/v1/journey/position    ###########
\subsection{Hinzuf�gen eines Fahrzeuges}
F�gt ein neues Fahrzeug der Datenbank hinzu.

\subsubsection{HTTP-Method}
POST

\subsubsection{Resource}
/api/v1/vehicle

\subsubsection{Payload}
\begin{lstlisting}
{
  "ref": "<uniqueId>",
  "routeId": <routeId>,
  "time": <timeStamp>,
  "position": {
    "lng": <longitude>,
    "lat": <latitude>
  }
}
\end{lstlisting}


\subsubsection{Response}
\begin{lstlisting}
<id>
\end{lstlisting}

\subsubsection{HTTP status codes}
\begin{itemize}
	\item 201: Resource erstellt
	\item 500: Bad payload
\end{itemize}

\subsubsection{Beispiel Request}
POST \url{http://domain.com/api/v1/position}

\begin{lstlisting}
{
  "ref": "636c81cc2361acd7",
  "routeId": 67,
  "time": 1513596185100,
  "position": {
    "lng": 52.3453,
    "lat": 13.53234
  }
}
\end{lstlisting}

\subsubsection{Beispiel Response}
\begin{lstlisting}
2
\end{lstlisting}


% ##########    PUT  /api/v1/journey/position    ###########
\subsection{Update eines Fahrzeuges}
Aktualisiert die Daten ein neues Fahrzeuges int der Datenbank hinzu.

\subsubsection{HTTP-Method}
PUT

\subsubsection{Resource}
/api/v1/vehicle

\subsubsection{Payload}
\begin{lstlisting}
{
  "ref": "<uniqueId>",
  "routeId": <routeId>,
  "time": <timeStamp>,
  "position": {
    "lng": <longitude>,
    "lat": <latitude>
  }
}
\end{lstlisting}

\subsubsection{HTTP status codes}
\begin{itemize}
	\item 201: Resource erstellt
	\item 500: Bad payload
\end{itemize}

\subsubsection{Beispiel Request}
POST \url{http://domain.com/api/v1/position}

\begin{lstlisting}
{
  "ref": "636c81cc2361acd7",
  "routeId": 67,
  "time": 1513596185100,
  "position": {
    "lng": 52.3453,
    "lat": 13.53234
  }
}
\end{lstlisting}

